\documentclass[10pt]{article}
% \def\StudentVersion{}
\usepackage{../../common}
\makeatletter

\def\LecStr{Alexander Rush}
\def\LecNum{1}
\def\LecTitle{Lectures Notes on Search}
\def\LecDate{}

\def\Graph{\path node(A)[draw, initial, state] at (-2, 1) {A};
    \path node(B)[draw, state] at (-1, 3) {B};
    \path node(C)[draw, state, accepting] at (4, 2) {C};
    \path node(D)[draw, state] at (1, 1) {D};
    \path node(E)[draw, state] at (2, 3) {E};
    \path[draw] (A) --node[xshift=-0.2cm]{2} (B); 
    \path[draw] (B) --node[yshift=0.2cm]{4} (E); 
    \path[draw] (A) --node[yshift=0.2cm]{3} (D); 
    \path[draw] (A) --node[yshift=0.2cm]{5} (E); 
    \path[draw] (D) --node[yshift=0.2cm]{4} (C); 
    \path[draw] (E) --node[yshift=0.2cm]{4} (C); 
}

\def\Words{
  \matrix(dict)[matrix of nodes, ampersand replacement=\&]{
    Mary \& golpeo \& la \& bruja \& verde \\
    ~\\
    ~\\
    Mary \& slapped \& the \& green \& witch \\ };
}

\begin{document}
\MakeScribeTop{}

\section{Board 0}

Richard Karp

Harvard Undergraduate 1955, Ph.D 1959, APplied math

Turing paper 1950

Won the Turing Award in 1985

Among many other papers Held-Karp 1970

\section{Board 1}

Travel(l)ing Salesman Problem

\begin{center}
\begin{tabularx}{\linewidth}{llX}
  \toprule
  Name  & Type & Description \\
  \midrule
\\
 State &  & \censor{}  \\\\
 Action &   & \censor{} \\\\
 Initial & & \censor{} \\\\
 Goal & & \censor{} \\\\
 Cost & & \\\\
 \bottomrule
\end{tabularx}
\end{center}

Canonical example of an NP-Hard Problem

Tremendously important in all types of applications

\section{Board 2}

What is a heuristic for this problem 

Held and Karp 1970...

Admissable heuristic is Minimum spanning tree. 

Can calculate very fast (show example)

Admissable. And you can show it is consistent.

\section{Board 3}

PacMan is TSP! 


% \section{Board 4}

% Translation is TSP

\end{document}

